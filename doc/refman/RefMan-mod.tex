\section{Module system
\index{Modules}
\label{section:Modules}}

The module system provides a way of packaging related elements
together, as well as a mean of massive abstraction.

\begin{figure}[t]
\begin{centerframe}
\begin{tabular}{rcl}
{\modtype}  & ::= & {\qualid} \\
 & $|$ & {\modtype} \texttt{ with Definition }{\qualid} := {\term} \\
 & $|$ & {\modtype} \texttt{ with Module }{\qualid} := {\qualid} \\
 & $|$ & {\qualid} \nelist{\qualid}{}\\
 & $|$ & $!${\qualid} \nelist{\qualid}{}\\
 &&\\

{\onemodbinding}  & ::= & {\tt ( [Import|Export] \nelist{\ident}{} : {\modtype} )}\\
 &&\\

{\modbindings} & ::= & \nelist{\onemodbinding}{}\\
 &&\\

{\modexpr} & ::= & \nelist{\qualid}{} \\
           & $|$ & $!$\nelist{\qualid}{}
\end{tabular}
\end{centerframe}
\caption{Syntax of modules}
\end{figure}

In the syntax of module application, the $!$ prefix indicates that
any {\tt Inline} directive in the type of the functor arguments
will be ignored (see \ref{Inline} below).

\subsection{\tt Module {\ident}
\comindex{Module}}

This command is used to start an interactive module named {\ident}.

\begin{Variants}

\item{\tt Module {\ident} {\modbindings}}

  Starts an interactive functor with parameters given by {\modbindings}.

\item{\tt Module {\ident} \verb.:. {\modtype}}

  Starts an interactive module specifying its module type. 

\item{\tt Module {\ident} {\modbindings} \verb.:. {\modtype}}

  Starts an interactive functor with parameters given by
  {\modbindings}, and output module type {\modtype}.

\item{\tt Module {\ident} \verb.<:. {\modtype$_1$} \verb.<:. $\ldots$ \verb.<:.{ \modtype$_n$}}

  Starts an interactive module satisfying each {\modtype$_i$}. 

\item{\tt Module {\ident} {\modbindings} \verb.<:. {\modtype$_1$} \verb.<:. $\ldots$ \verb.<:. {\modtype$_n$}}

  Starts an interactive functor with parameters given by
  {\modbindings}. The output module type is verified against each
  module type {\modtype$_i$}.

\item\texttt{Module [Import|Export]}

  Behaves like \texttt{Module}, but automatically imports or exports
  the module.

\end{Variants}
\subsubsection{Reserved commands inside an interactive module:
\comindex{Include}}
\begin{enumerate}
\item {\tt Include {\module}}

  Includes the content of {\module} in the current interactive
 module. Here {\module} can be a module expresssion or a module type
 expression. If {\module} is a high-order module or module type
 expression then the system tries to instanciate {\module}
 by the current interactive module.

\item {\tt Include {\module$_1$} \verb.<+. $\ldots$ \verb.<+. {\module$_n$}} 

is a shortcut for {\tt Include {\module$_1$}}  $\ldots$ {\tt Include {\module$_n$}}
\end{enumerate}
\subsection{\tt End {\ident}
\comindex{End}}

This command closes the interactive module {\ident}. If the module type
was given the content of the module is matched against it and an error
is signaled if the matching fails. If the module is basic (is not a
functor) its components (constants, inductive types, submodules etc) are
now available through the dot notation.

\begin{ErrMsgs}
\item \errindex{No such label {\ident}}
\item \errindex{Signature components for label {\ident} do not match}
\item \errindex{This is not the last opened module}
\end{ErrMsgs}


\subsection{\tt Module {\ident} := {\modexpr}
\comindex{Module}}

This command defines the module identifier {\ident} to be equal to
{\modexpr}. 

\begin{Variants}
\item{\tt Module {\ident} {\modbindings} := {\modexpr}}

 Defines a functor with parameters given by {\modbindings} and body {\modexpr}.

% Particular cases of the next 2 items
%\item{\tt Module {\ident} \verb.:. {\modtype} := {\modexpr}}
%
%  Defines a module with body {\modexpr} and interface {\modtype}.
%\item{\tt Module {\ident} \verb.<:. {\modtype} := {\modexpr}}
%
%  Defines a module with body {\modexpr}, satisfying {\modtype}.

\item{\tt Module {\ident} {\modbindings} \verb.:. {\modtype} :=
    {\modexpr}}

  Defines a functor with parameters given by {\modbindings} (possibly none),
  and output module type {\modtype}, with body {\modexpr}. 

\item{\tt Module {\ident} {\modbindings} \verb.<:.  {\modtype$_1$} \verb.<:. $\ldots$ \verb.<:. {\modtype$_n$}:=
    {\modexpr}}

  Defines a functor with parameters given by {\modbindings} (possibly none) 
  with body {\modexpr}. The body is checked against each {\modtype$_i$}.

\item{\tt Module {\ident} {\modbindings} := {\modexpr$_1$} \verb.<+. $\ldots$ \verb.<+. {\modexpr$_n$}}

  is equivalent to an interactive module where each {\modexpr$_i$} are included.

\end{Variants}

\subsection{\tt Module Type {\ident}
\comindex{Module Type}}

This command is used to start an interactive module type {\ident}.

\begin{Variants}

\item{\tt Module Type {\ident} {\modbindings}}

  Starts an interactive functor type with parameters given by {\modbindings}.

\end{Variants}
\subsubsection{Reserved commands inside an interactive module type:
\comindex{Include}\comindex{Inline}}
\label{Inline}
\begin{enumerate}
\item {\tt Include {\module}}

 Same as {\tt Include} inside a module.

\item {\tt Include {\module$_1$} \verb.<+. $\ldots$ \verb.<+. {\module$_n$}} 

is a shortcut for {\tt Include {\module$_1$}}  $\ldots$ {\tt Include {\module$_n$}}

\item {\tt {\assumptionkeyword} Inline {\assums} }

 The instance of this assumption will be automatically expanded at functor
 application, except when this functor application is prefixed by a $!$ annotation.
\end{enumerate}
\subsection{\tt End {\ident}
\comindex{End}}

This command closes the interactive module type {\ident}.

\begin{ErrMsgs}
\item \errindex{This is not the last opened module type}
\end{ErrMsgs}

\subsection{\tt Module Type {\ident} := {\modtype}}

Defines a module type {\ident} equal to {\modtype}.

\begin{Variants}
\item {\tt Module Type {\ident} {\modbindings} := {\modtype}}

  Defines a functor type {\ident} specifying functors taking arguments
  {\modbindings} and returning {\modtype}.

\item{\tt Module Type {\ident} {\modbindings} := {\modtype$_1$} \verb.<+. $\ldots$ \verb.<+. {\modtype$_n$}}

  is equivalent to an interactive module type were each {\modtype$_i$} are included.

\end{Variants}

\subsection{\tt Declare Module {\ident} : {\modtype}
\comindex{Declare Module}}

Declares a module {\ident} of type {\modtype}.

\begin{Variants}

\item{\tt Declare Module {\ident} {\modbindings} \verb.:. {\modtype}}

  Declares a functor with parameters {\modbindings} and output module
  type {\modtype}.


\end{Variants}


\subsubsection{Example}

Let us define a simple module.
\begin{coq_example}
Module M.
  Definition T := nat.
  Definition x := 0.
  Definition y : bool.
    exact true.
  Defined.
End M.
\end{coq_example}
Inside a module one can define constants, prove theorems and do any
other things that can be done in the toplevel. Components of a closed
module can be accessed using the dot notation:
\begin{coq_example}
Print M.x.
\end{coq_example}
A simple module type:
\begin{coq_example}
Module Type SIG.
  Parameter T : Set.
  Parameter x : T.
End SIG.
\end{coq_example}

Now we can create a new module from \texttt{M}, giving it a less
precise specification: the \texttt{y} component is dropped as well
as the body of \texttt{x}.

\begin{coq_eval}
Set Printing Depth 50.
(********** The following is not correct and should produce **********)
(***************** Error: N.y not a defined object *******************)
\end{coq_eval}
\begin{coq_example}
Module N  :  SIG with Definition T := nat  :=  M.
Print N.T.
Print N.x.
Print N.y.
\end{coq_example}
\begin{coq_eval}
Reset N.
\end{coq_eval}

\noindent
The definition of \texttt{N} using the module type expression
\texttt{SIG with Definition T:=nat} is equivalent to the following
one:

\begin{coq_example*}
Module Type SIG'.
  Definition T : Set := nat.
  Parameter x : T.
End SIG'.
Module N : SIG' := M.
\end{coq_example*}
If we just want to be sure that the our implementation satisfies a
given module type without restricting the interface, we can use a
transparent constraint
\begin{coq_example}
Module P <: SIG := M.
Print P.y.
\end{coq_example}
Now let us create a functor, i.e. a parametric module
\begin{coq_example}
Module Two (X Y: SIG).
\end{coq_example}
\begin{coq_example*}
  Definition T := (X.T * Y.T)%type.
  Definition x := (X.x, Y.x).
\end{coq_example*}
\begin{coq_example}
End Two.
\end{coq_example}
and apply it to our modules and do some computations
\begin{coq_example}
Module Q := Two M N.
Eval compute in (fst Q.x + snd Q.x).
\end{coq_example}
In the end, let us define a module type with two sub-modules, sharing
some of the fields and give one of its possible implementations:
\begin{coq_example}
Module Type SIG2.
  Declare Module M1 : SIG.
  Module M2 <: SIG.
    Definition T := M1.T.
    Parameter x : T.
  End M2.
End SIG2.
\end{coq_example}
\begin{coq_example*}
Module Mod <: SIG2.
  Module M1.
    Definition T := nat.
    Definition x := 1.
  End M1.
  Module M2 := M.
\end{coq_example*}
\begin{coq_example}
End Mod.
\end{coq_example}
Notice that \texttt{M} is a correct body for the component \texttt{M2}
since its \texttt{T} component is equal \texttt{nat} and hence
\texttt{M1.T} as specified.
\begin{coq_eval}
Reset Initial.
\end{coq_eval}

\begin{Remarks}
\item Modules and module types can be nested components of each other.
\item One can have sections inside a module or a module type, but
  not a module or a module type inside a section.
\item Commands like \texttt{Hint} or \texttt{Notation} can
  also appear inside modules and module types. Note that in case of a
  module definition like:

    \smallskip
    \noindent
    {\tt Module N : SIG := M.} 
    \smallskip

    or

    \smallskip
    {\tt Module N : SIG.\\
      \ \ \dots\\
      End N.}
    \smallskip 
    
    hints and the like valid for \texttt{N} are not those defined in
    \texttt{M} (or the module body) but the ones defined in
    \texttt{SIG}.

\end{Remarks}

\subsection{\tt Import {\qualid}
\comindex{Import}
\label{Import}}

If {\qualid} denotes a valid basic module (i.e. its module type is a
signature), makes its components available by their short names.

Example:

\begin{coq_example}
Module Mod.
\end{coq_example}
\begin{coq_example}
  Definition T:=nat.
  Check T.
\end{coq_example}
\begin{coq_example}
End Mod.
Check Mod.T.
Check T. (* Incorrect ! *)
Import Mod.
Check T. (* Now correct *)
\end{coq_example}
\begin{coq_eval}
Reset Mod.
\end{coq_eval}

Some features defined in modules are activated only when a module is
imported. This is for instance the case of notations (see
Section~\ref{Notation}).

Declarations made with the {\tt Local} flag are never imported by the
{\tt Import} command. Such declarations are only accessible through their
fully qualified name.

Example:

\begin{coq_example}
Module A.
Module B.
Local Definition T := nat.
End B.
End A.
Import A.
Check B.T.
\end{coq_example}

\begin{Variants}
\item{\tt Export {\qualid}}\comindex{Export}

  When the module containing the command {\tt Export {\qualid}} is
  imported, {\qualid} is imported as well.
\end{Variants}

\begin{ErrMsgs}
  \item \errindexbis{{\qualid} is not a module}{is not a module}
% this error is impossible in the import command
%  \item \errindex{Cannot mask the absolute name {\qualid} !}
\end{ErrMsgs}

\begin{Warnings}
  \item Warning: Trying to mask the absolute name {\qualid} !
\end{Warnings}

\subsection{\tt Print Module {\ident}
\comindex{Print Module}}

Prints the module type and (optionally) the body of the module {\ident}.

\subsection{\tt Print Module Type {\ident}
\comindex{Print Module Type}}

Prints the module type corresponding to {\ident}.

\subsection{\tt Locate Module {\qualid}
\comindex{Locate Module}}

Prints the full name of the module {\qualid}.


%%% Local Variables: 
%%% mode: latex
%%% TeX-master: "Reference-Manual"
%%% End: 
